% LaTeX file for resume 
% This file uses the resume document class (res.cls)

\documentclass[line]{res} 
\usepackage{helvetica} % uses helvetica postscript font (download helvetica.sty)
\usepackage{hyperref} % useful for links

%%%\usepackage{newcent}   % uses new century schoolbook postscript font 

\begin{document} 
\urlstyle{rm}

\name{David C. Gemperline}

\address{2354 Colfax Ln, Indianapolis, IN 46260}
\address{dcgemperline@gmail.com 608-886-8571}


\begin{resume}

\section{HIGHLIGHTS}
Software development in C\#, Perl, and Java \\
\vspace{0.1in}	
Extensive experience in proteomic sample preparation and development of quantitative proteomic analysis software for high resolution mass spectrometry \\
\vspace{0.1in}	
Experience with independent biological research designs (included next generation sequencing, genomic, proteomic, and phenotypic data analysis) and broad knowledge of computational data anlysis techniques \\
\vspace{0.1in}	
Knowledge of Solid software design principles and algorithmic complexity

\section{EDUCATION}          
\textbf{Doctor of Philosphy in Genetics}, 2016 \\
University of Wisconsin-Madison, Madison, WI

\textbf{Bachelor of Arts in Chemistry and Biology}, 2009 \\
Carthage College, Kenosha, WI \\
\textbf{GPA: 3.92/4.00}, \textit{Summa Cum Laude}

\section{TECHNICAL SKILLS}
Project lead developing a novel label-free quantitative proteomics pipeline: I developed an open-source C\# command line application with a WinForms front-end to extend user friendliness.
\vspace{-0.2in}	
\begin{tabbing}
   \hspace{3.0in}\= \kill % set up one tab position and kill blank line after tabbing
   Language experience in R, C\#, Perl, Java \> GitHub \url{http://dcgemperline.github.io} \\
\end{tabbing} \vspace{-20pt}
\textbf{Software Development Tools}: Visual Studio 2017, Resharper, Git, SVN, EMACS \\
\textbf{Knowledge}: Object-Oriented and functional programming design, Test-Driven Development (TDD), OS Experience with Windows and Linux \\
\textbf{Continued Profesional Development}: Coursera - Algorithms Part I and Machine Learning by Andrew Ng

\section{RELEVANT SKILLS}
Leadership and mentoring experience \\
Experience working on interdisciplinary research teams \\
Excellent oral and written communication skills: \textbf{Invited speaker at an international Gordon Research Conference on Post-Translational Modifications}

\section{EXPERIENCE}
   \vspace{-0.1in}	
   \begin{tabbing}
   \hspace{2.55in}\= \hspace{2.30in}\= \kill % set up two tab positions
    {\bf Postdoc Immunology - Bioinformatics} \>Eli Lilly and Company \> Oct 2016 - Present\\
                             \>Indianapolis, IN
   \end{tabbing}\vspace{-20pt}      % suppress blank line after tabbing
   Analyzed and integrated multi-omic clinical data for patient stratification
   \vspace{-10pt}
   \begin{tabbing}
   \hspace{2.3in}\= \hspace{2.45in}\= \kill % set up two tab positions
    {\bf Research Assistant} \>University of Wisconsin-Madison \> Aug 2009 - Oct 2016 \\
                          \>Madison, WI
   \end{tabbing}\vspace{-10pt}
   \begin{itemize}                     
     \item Performed data analysis and custom software development for biological data analysis pipelines with a focus on mass spectrometry-based label-free quantitative proteomics
     \item Laboratory expert on using and configuring next generation sequencing (NGS) data analysis software from Linux Command line
     \item Generation of transgenic plants expressing tagged epitope variants
   \end{itemize}
 
\end{resume}
\end{document}

%%% Local Variables:
%%% mode: latex
%%% TeX-master: t
%%% End:
